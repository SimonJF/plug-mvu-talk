\usepackage{amsmath, amssymb, stmaryrd, mathpartir, commath, url, longtable,hyperref,tabularx}

\usepackage{forloop, microtype}
\usepackage{wasysym}
%\usepackage[dvipsnames]{xcolor}
\usepackage{csquotes}
%\usepackage{mathptmx}
\usepackage{listings}
\hypersetup{colorlinks=true,citecolor=blue}
\RequirePackage{subcaption}
\captionsetup{compatibility=false}
\usepackage{graphicx}
\usepackage{xspace}
\usepackage{dirtytalk}
\usepackage{xparse}% http://ctan.org/pkg/xparse
\usepackage{etoolbox}% http://ctan.org/pkg/etoolbox
\newcommand{\lambdacalc}{\ensuremath{\lambda}-calculus\xspace}
\newenvironment{fake}[1]{\par\vspace{3pt}\noindent\textbf{#1}\itshape}{\normalfont\ignorespacesafterend\vspace{3pt}\par}

\usepackage{float}
\floatstyle{boxed}
\restylefloat{figure}
\usepackage{verbatim}

\newcommand{\backupbegin}{
   \newcounter{finalframe}
   \setcounter{finalframe}{\value{framenumber}}
}
\newcommand{\backupend}{
   \setcounter{framenumber}{\value{finalframe}}
}

\setbeamercovered{%
  again covered={\opaqueness<1->{15}}}



% TIKZ STUFF


\usepackage{amsmath,verbatim}

\lstdefinelanguage{Links}{%
  morekeywords={typename, fun, linfun, op, var, if, this, true, false, else, case, switch, handle,
    handler, shallowhandler, open, do, sig, new, send, receive, spawnAt, spawn,
module, request, accept, try, as, in, otherwise, catch, offer, select, raise,
fork, spawnClient, cancel, catch},%
  sensitive=t, %
  comment=[l]{\#\ },%
  escapeinside={(*}{*)},%
  morestring=[d]{"},%
  keywordstyle=\color{blue},
  showstringspaces=false
  %frame = single
 }

% Links style
\lstset{
  basicstyle=\linespread{1.3}\ttfamily\large,
  keywordstyle=\bfseries,
  language=Links,
  backgroundcolor=\color{white}
}


\usepackage{expl3,xparse}
\makeatletter
\let\old@lstKV@SwitchCases\lstKV@SwitchCases
\def\lstKV@SwitchCases#1#2#3{}
\makeatother
\usepackage{lstlinebgrd}
\makeatletter
\let\lstKV@SwitchCases\old@lstKV@SwitchCases

\lst@Key{numbers}{none}{%
    \def\lst@PlaceNumber{\lst@linebgrd}%
    \lstKV@SwitchCases{#1}%
    {none:\\%
     left:\def\lst@PlaceNumber{\llap{\normalfont
                \lst@numberstyle{\thelstnumber}\kern\lst@numbersep}\lst@linebgrd}\\%
     right:\def\lst@PlaceNumber{\rlap{\normalfont
                \kern\linewidth \kern\lst@numbersep
                \lst@numberstyle{\thelstnumber}}\lst@linebgrd}%
    }{\PackageError{Listings}{Numbers #1 unknown}\@ehc}}
\makeatother

\ExplSyntaxOn
\NewDocumentCommand \lstcolorlines { O{green} m }
{
  \clist_if_in:nVTF { #2 } { \the\value{lstnumber} }{ \color{#1} }{\color{white}}
}
\ExplSyntaxOff
\newcommand{\textl}{\text{\lightning}}


\newcommand{\calcwd}[1]{\mathbf{\mathbf{#1}}}
\newcommand{\one}{\mathbf{1}}
\newcommand{\lightningnu}[1]{\nu #1\lightning}
\newcommand{\gvqueue}[4]{#1(#2){\leftrightsquigarrow} #3(#4)}
\newcommand{\mkwd}[1]{\ensuremath{\mathsf{#1}}}
\newcommand{\gvsend}[2]{\calcwd{send} \: #1 \: #2}
\newcommand{\gvrecv}[1]{\calcwd{receive} \: #1}
\newcommand{\gvcancel}[1]{\calcwd{cancel} \app #1}
\newcommand{\gvclose}[1]{\calcwd{close} \app #1}
\newcommand{\gvcancelup}[1]{\calcwd{cancel} \app #1}
\newcommand{\gvout}[2]{!{#1}.{#2}}
\newcommand{\gvin}[2]{?{#1}.{#2}}
\newcommand{\gvend}{\mathsf{End}\xspace}
\newcommand{\gvfork}[1]{\calcwd{fork} \, #1}
\newcommand{\app}{\:}
\newcommand{\inl}[1]{\calcwd{inl} \app #1}
\newcommand{\inr}[1]{\calcwd{inr} \app #1}
\newcommand{\gvcase}[2]{\calcwd{case} \: #1 \: \calcwd{of} \: \{ #2 \}  }
\newcommand{\gvlet}[3]{\calcwd{let} \: #1 = #2 \: \calcwd{in} \: #3}
\newcommand{\tryasinotherwise}[4]{\calcwd{try} \: #1 \: \calcwd{as} \: #2 \: \calcwd{in} \: #3 \: \calcwd{otherwise} \: #4}
\newcommand{\raiseexn}{\calcwd{raise}\xspace}
\newcommand{\un}[1]{\mkwd{un}(#1)}
\newcommand{\gvdual}[1]{\overline{#1}}
%\newcommand{\compat}{\asymp}
\newcommand{\evalperhaps}{\eval^?}
\newcommand{\evalstar}{\eval^*}
\newcommand{\scancel}[1]{\text{\lightning} #1}
\newcommand{\scancelmv}[1]{#1 \lightning}
\newcommand{\config}[1]{\mathcal{#1}}
\newcommand{\seq}[1]{\overrightarrow{#1}}
\newcommand{\fv}[1]{\mathsf{fv}(#1)}
%\newcommand{\fcv}[1]{\mathsf{fcv}(#1)}
\newcommand{\fvs}[1]{\mathsf{fvs}(#1)}
\newcommand{\fn}[1]{\mathsf{fn}(#1)}
\newcommand{\fcvs}[1]{\fn{#1}}
\newcommand{\chansharp}[1]{#1^{\sharp}}
\newcommand{\chanflat}[1]{#1^{\flat}}
\newcommand{\without}{/}
\newcommand{\equivceval}{\equiv \longrightarrow_{\textsf{C}} \equiv}
\newcommand{\cevalthick}{\Longrightarrow_{\textsf{C}}}
\newcommand{\ceval}{\quad \longrightarrow \quad}
\newcommand{\teval}{\longrightarrow_{\textsf{M}}}
\newcommand{\eps}[1]{\mkwd{eps}(#1)}
\newcommand{\affected}[1]{\mkwd{affected}(#1)}
\newcommand{\halt}{\calcwd{halt}\xspace}
\newcommand{\bcirc}{\bullet}
\newcommand{\wcirc}{\circ}
\newcommand{\disable}[1]{\mkwd{disable}(#1)}

\newcommand{\blocked}[2]{\mkwd{blocked}(#1, #2)}
\newcommand{\depends}[3]{\mkwd{depends}(#1, #2, #3)}
\newcommand{\deadlocked}[1]{\mkwd{deadlocked}(#1)}

\newcommand{\Ceval}{\Longrightarrow_{\textsf{C}}}

\newcommand{\dom}[1]{\mkwd{dom}(#1)}

\newenvironment{remark}{\paragraph{Remark.}}{\hfill \tiny{$\square$}}
%\newtheorem{remark}{Remark}
%\newtheorem{lemma}{Lemma}
%\newtheorem{proposition}{Proposition}
%\newtheorem{corollary}{Corollary}
%\newtheorem{definition}{Definition}


\newcommand{\defeq}{\triangleq}

\newcommand{\todo}[1]{{\noindent\small\color{red} \framebox{\parbox{\dimexpr\linewidth-2\fboxsep-2\fboxrule}{\textbf{TODO:} #1}}}}
\newenvironment{subcase}[1]
  {\textbf{Subcase #1} \hfill \\
  \begin{adjustwidth}{0.25cm}{}}
  {\end{adjustwidth}}

\newenvironment{proofcase}[1]
  {\totheleft{\textbf{Case \textsc{#1}}}}
  {}
\newcommand{\lto}{\multimap}
\newcommand{\proofstep}[1]{
  \begin{compactitem}
  \item #1
  \end{compactitem}
}

\newcommand{\names}[1]{\mkwd{names}(#1)}

\arraycolsep=1pt%\def\arraystretch{2.2}

\newcommand{\transl}[1]{\llbracket #1 \rrbracket}
\newcommand{\stranslateb}[1]{\mathcal{S}\llparenthesis #1 \rrparenthesis}
\newcommand{\ttranslateb}[1]{\mathcal{T}\llparenthesis #1 \rrparenthesis}
\newcommand{\etranslateb}[1]{\mathcal{E}\llparenthesis #1 \rrparenthesis}
\newcommand{\ptranslateb}[1]{\mathcal{P}\llparenthesis #1 \rrparenthesis}

\newcommand{\gvoutone}[1]{{!}#1}
\newcommand{\gvinone}[1]{{?}#1}
\newcommand{\letintwo}[2]{\calcwd{let} \: #1 = #2 \: \calcwd{in}}
\newcommand{\totheleft}[1]{\begin{flushleft}#1\end{flushleft}}
\usepackage[T1]{fontenc}
\usepackage[scaled=0.85]{beramono}
\newcommand\doubleplus{+\kern-1.3ex+\kern0.8ex}
\newcommand{\metadef}[1]{\calcwd{#1}}


\newcommand{\runtimechan}[2]{\mkwd{Channel}(#1, #2)}
\newcommand{\closure}[2]{\langle #1, #2 \rangle}
%\newcommand{\boldleft}{\textbf{\{ }}
%\newcommand{\boldright}{\textbf{ \}}}
\newcommand{\contained}[1]{\metadef{Channels}(#1)}

\newcommand{\ba}{\begin{array}}
\newcommand{\ea}{\end{array}}

\newcommand{\bl}{\ba[t]{@{}l@{}}}
\newcommand{\el}{\ea}


\newcommand{\triestosend}[4]{{#1} \xRightarrow[#4]{#2} {#3}}

\newcommand{\notsmall}{}

\newcommand{\oftype}{\,{:}\,}

\newcommand{\sendpair}[2]{\calcwd{send} \: (#1, #2)}

%\renewcommand{\paragraph}[1]{\textbf{\textit{#1}}}
\usepackage{tikz-cd}

\tikzstyle{arrow} = [thick,->]
\tikzstyle{diagnode} = [rectangle, rounded corners, minimum width=8cm, minimum
height=1cm,text centered, text width=8cm, draw=black, anchor=center, fill=red!30]


\newcommand{\exnty}{\mkwd{Exn}\xspace}
\newcommand{\raiseexnP}[1]{\calcwd{raise} \: #1}
\newcommand{\tryinunless}[4]{\calcwd{try} \: #1 \: \calcwd{as} \: #2 \:
    \calcwd{in} \: #3 \: \calcwd{unless} \: #4}

\newcommand{\handled}[1]{\mkwd{Handled}(#1)}

\newcommand{\highlight}[1]{{\color{colorredorange} #1} }
\newcommand{\highlightpink}[1]{{\color{colorpink} #1} }
\newcommand{\highlightwarning}[1]{{\color{colorlightpink} #1} }
\newcommand{\midspace}{\; \mid{} \; }
\setlength\tabcolsep{1.5em}

\newcommand{\lact}{\lambda_{\text{act}}\xspace}
\newcommand{\lch}{\lambda_{\text{ch}}\xspace}


\lstdefinelanguage{JavaScript}{
  keywords={typeof, new, true, false, catch, function, return, null, catch, switch, var, if, in, while, do, else, case, break},
  keywordstyle=\color{blue}\bfseries,
  ndkeywords={class, export, boolean, throw, implements, import, this},
  ndkeywordstyle=\color{darkgray}\bfseries,
  identifierstyle=\color{black},
  sensitive=false,
  comment=[l]{//},
  morecomment=[s]{/*}{*/},
  commentstyle=\color{purple}\ttfamily,
  stringstyle=\color{red}\ttfamily,
  morestring=[b]',
  morestring=[b]"
}

\lstset{
   language=JavaScript,
   backgroundcolor=\color{lightgray},
   extendedchars=true,
   basicstyle=\footnotesize\ttfamily,
   showstringspaces=false,
   showspaces=false,
   numbers=left,
   numberstyle=\footnotesize,
   numbersep=9pt,
   tabsize=2,
   breaklines=true,
   showtabs=false,
   captionpos=b
}

