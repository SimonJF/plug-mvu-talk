\documentclass[11.5pt, aspectratio=169]{beamer}
\input{preamble}
\usepackage{config/presento}
\usepackage{framed}
\usepackage{stmaryrd}
\usepackage{amsmath}
\usepackage{wasysym}
% Presento style file
\usepackage{float,lipsum}
\floatstyle{boxed}
\usepackage{mathpartir}
% custom command and packages
\input{config/custom-command}
\usepackage{verbatim}
% Information
\title{Model-View-Update, and Beyond!}
\subtitle{Adventures in Formalising and Extending the Elm Architecture}
\author{Simon Fowler}
\institute{University of Edinburgh}
\date{12th March 2019}

\begin{document}

% Title page
\begin{frame}[plain]
\titlepage


\hfill
\vspace{-1em}
$
\renewcommand*{\arraystretch}{1.8}
\begin{array}{r}
   \includegraphics[height=0.7cm, keepaspectratio]{images/logos/inf_uoe.png} \\
\end{array}
$
\end{frame}
\begin{frame}{Links: Web Programming without Tiers}

  \centering
  \includegraphics[width=0.5\paperwidth]{images/3tier.pdf}
  \vspace{1em}

  \begin{fullpageitemize}
    \itemR \textbf{Idea}: Uniform language for ``tierless'' web programming: client, server, DB code all in same language (Cooper et al., 2006)
    \itemR Statically-typed, ML-inspired, impure functional language
    \itemR With \emph{lots} of cool research features
  \end{fullpageitemize}
  \vspace{1em}
\end{frame}

\begin{frame}{Minimal Example: A Box and a Label}
  \begin{center}
    \includegraphics[scale=0.5]{images/label.png}
  \end{center}

  \begin{fullpageitemize}
  \itemR Very small illustrative example
  \itemR Text contained in a \texttt{div} should reflect value of a form input box, reversed
  \end{fullpageitemize}
\end{frame}


\begin{frame}[fragile]{Vanilla Links Implementation}
%fun getProp(id, propName) { ... }
%fun setProp(id, propName, newVal) { ... }
  \begin{lstlisting}[language=links]
fun mainPage() {
  fun handleEvents() {
    receive { case UpdateLabel ->
      var value = getProp("toReverse", "value");
      setProp("label", "innerHTML", reverseString(value));
      handleEvents()
    }
  }
  var evtHandler = spawnClient { handleEvents() };
  page
    <html>
      <body>
        <form>
          <input type="text" id="toReverse"
            l:onkeyup="{evtHandler ! UpdateLabel}"></input>
        </form>
        <div id="label"></div>
      </body>
    </html>
}
  \end{lstlisting}
\end{frame}

\begin{frame}{Vanilla Links Model}

  \begin{center}
     % TODO: Diagram (TiKZ?)
    \begin{tikzpicture}[node distance=1.5cm]
      \node (0) [diagnode] { Event invoked };
      \node (1) [diagnode, below of=0]{ Message sent to handler };
      \node (2) [diagnode, below of=1]{ Process reads data from DOM };
      \node (3) [diagnode, below of=2]{ Process updates data in DOM };

      \draw [arrow] (0) -- (1);
      \draw [arrow] (1) -- (2);
      \draw [arrow] (2) -- (3);
    \end{tikzpicture}
  \end{center}

  \begin{fullpageitemize}
  \itemR \textbf{Depressingly} imperative
    \begin{itemize}
  \itemR DOM is essentially one big chunk of mutable state
  \itemR Explicit retrieval from / mutation of DOM elements
  \end{itemize}
  \end{fullpageitemize}
\end{frame}


\framecard{{\color{white}\bigtext{Model-View-Update, or The Elm Model}}}

\begin{frame}{The Elm Programming Language}

  \begin{center}
    \includegraphics[width=0.5\textwidth]{images/ElmLogo.png}
  \end{center}

  \begin{fullpageitemize}
  \itemR Statically-typed, Haskell-esque, purely-functional programming language, geared towards web programming
  \itemR Originally based on applicative functional reactive programming
    \begin{itemize}
      \itemR \ldots however the designers later (correctly) realised that message passing was a better fit
      \itemR Pioneers the \emph{model-view-update} pattern
    \end{itemize}
  \end{fullpageitemize}

\end{frame}

\begin{frame}[fragile]{Links-Elm: A Box and a Label}

\begin{lstlisting}[language=Links]
typename Model = (contents: String);
typename Message = [| UpdateBox: String |];

sig updt : (Message, Model) ~%~> Model
fun updt(msg, model) {
  switch (msg) { case UpdateBox(newStr) -> (contents = newStr) }
}
fun view(model) {
  var a0 = MvuAttrs.empty; var h0 = MvuHTML.empty;
  div(a0,
    form(a0,
      input(type("text") +@ onKeyUp(fun(str) { UpdateBox(str) }), h0)) +*
    div(a0, textNode(reverseString(model.contents))))
}
fun mainPage() {
  Mvu.runSimple("placeholder", (contents=""), view, updt);
  page <html><body><div id="placeholder"></div></body></html>
}
\end{lstlisting}
\end{frame}

\begin{frame}[fragile]{The Essence of the Elm Architecture}

  \begin{fullpageitemize}
  \itemR<1-> User-defined types:
    \begin{itemize}
      \itemR Model type \verb+Model+, containing application data
      \itemR Message type \verb+Msg+, describing messages produced by components
    \end{itemize}
    \vspace{0em}

  \itemR<2-> HTML type \verb+HTML(Msg)+: an HTML element producing a message of
    type \verb+Msg+

  \itemR<3-> A function, \verb+run+:
    \begin{verbatim}
      run: forall model, msg.
        (model,                          // Initial model
         (model) -> HTML(msg),           // View function
         (model, msg) ~> model)          // Update function
        ~> ()
    \end{verbatim}

    \vspace{0em}

  \itemR<4-> HTML generated by \verb+Render+ \emph{diffed} against current DOM, lightweight updates
  \itemR<5-> (Also \emph{subscriptions} which allow events like time, mouse movement
    to generate messages---omitted here for simplicity)
  \end{fullpageitemize}
\end{frame}

\begin{frame}[fragile]{Implementing the Elm Architecture in Links}

  \begin{minipage}{0.475\textwidth}
  \begin{lstlisting}[language=links]
fun evtLoop(model, render, updt) {
  receive {
    case msg ->
      var newModel = updt(msg, model);
      # Update DOM
      VDom.updateDom(
        render(newModel));
      # Loop with new model
      evtLoop(newModel, render, updt)
    }
}
\end{lstlisting}
\end{minipage}
~\hfill
\begin{minipage}{0.5\textwidth}
\begin{lstlisting}[language=JavaScript]
function _updateDom(doc) {
  var newTree = jsonToVtree(doc);
  var patches =
    diff(currentVDom, newTree);
  currentVDom = newTree;
  rootNode =
    patch(rootNode, patches);
}
\end{lstlisting}
\end{minipage}

  \begin{fullpageitemize}
  \itemR Implementation with Jake Browning, LFCS intern, over summer 2017
  \itemR Made use of typed actor-style concurrency in Links (since updated)
  \itemR Required implementation of JS FFI
  \end{fullpageitemize}
\end{frame}

\begin{frame}{Real applications!}

  \begin{center}
    \includegraphics[width=0.75\textwidth]{images/spls-reg.png}
  \end{center}

  \begin{itemize}
    \itemR Also TodoMVC, Pong, a few more\ldots
  \end{itemize}
\end{frame}


\begin{frame}{Two questions}
  \begin{fullpageitemize}
  \item {\Large \textbf{What would be a sensible formal model for MVU?}}
    \begin{itemize}
      \itemR Formal model would make the paradigm more precisely-defined and easier to implement in future
    \end{itemize}
  \item {\Large \textbf{Can we use MVU to write web applications with distributed session types?}}
    \begin{itemize}
      \itemR Writing client code using distributed session-typed channels in Links is painful: can we do better?
    \end{itemize}
  \end{fullpageitemize}
\end{frame}

\framecard{{\color{white}\bigtext{Formalising MVU}}}

\begin{frame}{Formalism by Example}

\end{frame}

\begin{frame}{Key concepts}

  \begin{fullpageitemize}
  \end{fullpageitemize}
\end{frame}

\begin{frame}{Syntax}
\end{frame}

\begin{frame}{Typing Judgements}
\end{frame}

\begin{frame}{Configurations}
\end{frame}

\begin{frame}{Results}
\end{frame}

\framecard{{\color{white}\bigtext{MVU + Distributed Session Types}}}

\begin{frame}{Session Types in Links}
\end{frame}

\begin{frame}{Distributed Session Types}
\end{frame}

\begin{frame}{Linearity meets MVU}
\end{frame}

\begin{frame}{Session-typed communication meets MVU}
\end{frame}

\begin{frame}{A nontrivial application}
\end{frame}

\begin{frame}{People seem interested\ldots}
\end{frame}

\framecard{{\color{white}\bigtext{Wrapping up}}}

\begin{frame}{The future}
  \begin{fullpageitemize}
    \item {\large \textbf{Formalising Extensions}}
    \item {\large \textbf{Small-step interpreter}}
  \end{fullpageitemize}
\end{frame}

\begin{frame}{Conclusion}
\end{frame}

\end{document}

